\documentclass[12pt,a4paper]{article}
\usepackage[utf8]{inputenc}
\usepackage{graphicx}
\usepackage[none]{hyphenat}
\usepackage{hyperref}
\usepackage{xcolor}

\setlength{\fboxrule}{2pt}

\begin{document}
	\begin{titlepage}
		
		\begin{center}
			\includegraphics[width=0.5\textwidth]{QUT.jpg}\\
			[0.03\textheight]  
			\Large\textbf{Bachelor of IT (Computer Science)}\\
			\Large\textbf{Assignment 2 - Client-Side React Application}\\
			\large\textbf{CAB230 - Web Computing}\\
			[0.02\textheight]
			\large\textsl{Dane Madsen}\\
			\large\textsl{n10983864@qut.edu.au}
		\end{center}
		
	\end{titlepage}
	\tableofcontents
	\newpage
	
	\section{Introduction}
		\subsection{Purpose and Description}
			The purpose of this react application is to collate and display information regarding movies, 
			and cast members to the user in a responsive and accessible manner. The application should 
			allow the user to search for movies by year or title and display the results in a list. 
			When a movie is selected, the application should display all the details of the movie, 
			including the year the movie was made, the plot of the movie and all the cast members 
			involved with the movie. The application should also allow the user to visit individual 
			pages of cast members where the user can view that cast members details, including their 
			birth year, death year and all the movies they have been involved with.\\
			\\
			It achieve this goal and to provide the user with the best experience I can, i have used a 
			number of advanced react features, including the use of react router, IntesectionObserver and 
			react-responsive-carousel. The use of these features has allowed me to create a 
			professional looking application that is both easy to use and responsive for the user.\\

			\begin{center}
				\fbox{\includegraphics[width=\textwidth]{Figure1.jpg}}
				\fbox{\includegraphics[width=\textwidth]{Figure2.jpg}}
			\end{center}
		
		\subsection{Completeness and Limitations}
			This implementation covers all the requirements of the assignment specification. 
			Navigation is handled using react router, controlled forms are used for all user input, and 
			every 10 minutes the refresh token is used to get a new access token. In addition to this, 
			the application uses ag-grid to display the search results in an infinitly scrolling grid 
			and react-responsive-carousel to display the highest scoring movies of all time on the 
			home (landing) page. On the person page the movies the person is involved in are displayed 
			in a grid with the IMDb Rating.\\

	\newpage

	\section{Use of End Points}
		The functionality for all the API endpoints is handled in API.js. This file contains all the 
		functions that are used to make requests to the API. By containing all the API functionality 
		in one file, it makes it easier to maintain and potentially update the application in the 
		future.\\

		\subsection{/movies/search}
			This endpoint is implemented as the getMovies function and is utilised by two 
			search forms at the top of the application. The first search form is for the user to search 
			for movies by title, and the second is for the user to search for movies by year. Both of 
			these forms are controlled forms and can be accessed from any page in the application. Doing 
			it this way removes an extra layer of complexity for the user and allows them to get into the 
			function of the application right from the start of	the application.\\

			\begin{center}
				\fbox{\includegraphics[width=\textwidth]{Figure3.jpg}}
			\end{center}

		\newpage

		\subsection{/movies/data/\{imdbID\}}
			This endpoint is implemented as the getMovie function and is utilised by the movie page to 
			get the details of the movie the user has selected. The movie page is accessed by clicking 
			on any of the movies in the search results. The movie page displays all the details of the 
			movie, including the title, year, box office earnings, runtime, genre, country of origin, 
			plot, cast and ratings. each cast member is a clickable link that will lead to that cast 
			members respective page.\\

			\begin{center}
				\fbox{\includegraphics[width=\textwidth]{Figure4.jpg}}
			\end{center}

		\newpage

		\subsection{/people/\{id\}}
			This endpoint is implemented as the getPerson function and is utilised by the person page 
			to get the details of the person the user has selected. The person page is accessed by 
			clicking on any of the cast members on the movie page. The person page displays all the 
			details of the person exposed by the API, including their name, birth year, death year, 
			all the movies they have been involved with. In a grid the name of each movie, the name 
			of the character the person played and the IMDb Rating is displayed for the users 
			convenience.\\

			\begin{center}
				\fbox{\includegraphics[width=\textwidth]{Figure5.jpg}}
			\end{center}

		\newpage

		\subsection{/user/register}
			This endpoint is implemented as the postRegister function and is utilised by the profile 
			page to register new users. The profile page is accessed by clicking on the profile button 
			in the navidation bar at the top of the page. 

			\begin{center}
				\fbox{\includegraphics[width=\textwidth]{Figure6.jpg}}
			\end{center}


		\newpage

		\subsection{/user/login}
			This endpoint is implemented as the postLogin function and is utilised by the profile page 
			to login existing users.

			\begin{center}
				\fbox{\includegraphics[width=\textwidth]{Figure7.jpg}}
			\end{center}

		\newpage

		\subsection{/user/refresh}
			This endpoint is implemented as the postRefresh function and is utilised in the back end 
			in App.js to refresh the tokens when the app first starts and every 10 minutes after. 
			This is done to ensure the user does not have to login every 10 minutes.\\
		
		\subsection{/user/logout}
			This endpoint is implemented as the postLogout function and is utilised by the profile 
			page to logout the user.\\
			


\end{document}